%% abtex2-modelo-glossarios.tex, v-1.9.3 laurocesar
%% Copyright 2012-2015 by abnTeX2 group at http://abntex2.googlecode.com/
%%
%% This work may be distributed and/or modified under the
%% conditions of the LaTeX Project Public License, either version 1.3
%% of this license or (at your option) any later version.
%% The latest version of this license is in
%%   http://www.latex-project.org/lppl.txt
%% and version 1.3 or later is part of all distributions of LaTeX
%% version 2005/12/01 or later.
%%
%% This work has the LPPL maintenance status `maintained'.
%%
%% The Current Maintainer of this work is the abnTeX2 team, led
%% by Lauro César Araujo. Further information are available on
%% http://abntex2.googlecode.com/
%%
%% This work consists of the files abtex2-modelo-glossarios.tex,
%% abntex2-modelo-include-comandos and abntex2-modelo-references.bib
%%
                     
% ------------------------------------------------------------------------
% ------------------------------------------------------------------------
% abnTeX2: Exemplo de glossários com o pacote glossaries e abntex2
% ------------------------------------------------------------------------
% ------------------------------------------------------------------------
 
\documentclass[
	% -- opções da classe memoir --
	12pt,				% tamanho da fonte
	openright,			% capítulos começam em pág ímpar (insere página vazia caso preciso)
	twoside,			% para impressão em verso e anverso. Oposto a oneside
	a4paper,			% tamanho do papel. 
	% -- opções da classe abntex2 --
	%chapter=TITLE,		% títulos de capítulos convertidos em letras maiúsculas
	%section=TITLE,		% títulos de seções convertidos em letras maiúsculas
	%subsection=TITLE,	% títulos de subseções convertidos em letras maiúsculas
	%subsubsection=TITLE,% títulos de subsubseções convertidos em letras maiúsculas
	% -- opções do pacote babel --
	english,			% idioma adicional para hifenização
	french,				% idioma adicional para hifenização
	spanish,			% idioma adicional para hifenização
	brazil,				% o último idioma é o principal do documento
	]{abntex2}
 
 
% ---
% PACOTES
% ---
 
% ---
% Pacotes fundamentais
% ---
\usepackage{lmodern}			% Usa a fonte Latin Modern			
\usepackage[T1]{fontenc}		% Selecao de codigos de fonte.
\usepackage[utf8]{inputenc}		% Codificacao do documento (conversão automática dos acentos)
\usepackage{indentfirst}		% Indenta o primeiro parágrafo de cada seção.
\usepackage{color}				% Controle das cores
\usepackage{graphicx}			% Inclusão de gráficos
\usepackage{microtype} 			% para melhorias de justificação
% ---

% ---
% Pacotes glossaries
% ---
\usepackage[subentrycounter,seeautonumberlist,nonumberlist=true]{glossaries}
% para usar o xindy ao invés do makeindex:
%\usepackage[xindy={language=portuguese},subentrycounter,seeautonumberlist,nonumberlist=true]{glossaries}
% ---

% ---
% Pacotes de citações
% ---
\usepackage[brazilian,hyperpageref]{backref}	 % Paginas com as citações na bibl
\usepackage[alf]{abntex2cite}	% Citações padrão ABNT
 
% ---
% Informações de dados para CAPA e FOLHA DE ROSTO
% ---
\titulo{Exemplo de uso de glossário com \abnTeX}
\autor{Equipe \abnTeX}
\local{Brasil}
\data{2015, v-1.9.3}
\orientador{Lauro César Araujo}
\coorientador{Equipe \abnTeX}
\instituicao{%
  Universidade do Brasil -- UBr
  \par
  Faculdade de Arquitetura da Informação
  \par
  Programa de Pós-Graduação}
\tipotrabalho{Tese (Doutorado)}
% O preambulo deve conter o tipo do trabalho, o objetivo,
% o nome da instituição e a área de concentração
\preambulo{Modelo de texto com glossário conforme as
as normas ABNT apresentado à comunidade de usuários \LaTeX.}
% ---
 
 
% ---
% Configurações de aparência do PDF final
 
% alterando o aspecto da cor azul
\definecolor{blue}{RGB}{41,5,195}
 
% informações do PDF
\makeatletter
\hypersetup{
     	%pagebackref=true,
		pdftitle={\@title}, 
		pdfauthor={\@author},
    	pdfsubject={\imprimirpreambulo},
	    pdfcreator={LaTeX with abnTeX2},
		pdfkeywords={abnt}{latex}{abntex}{abntex2}{glossários}, 
		colorlinks=true,       		% false: boxed links; true: colored links
    	linkcolor=blue,          	% color of internal links
    	citecolor=blue,        		% color of links to bibliography
    	filecolor=magenta,      		% color of file links
		urlcolor=blue,
		bookmarksdepth=4
}
\makeatother

% ---
 
% ---
% Espaçamentos entre linhas e parágrafos
% ---
 
% O tamanho do parágrafo é dado por:
\setlength{\parindent}{1.3cm}
 
% Controle do espaçamento entre um parágrafo e outro:
\setlength{\parskip}{0.2cm}  % tente também \onelineskip
 
% ---
% compila o indice
% ---
\makeindex
% ---
 
% ---
% GLOSSARIO
% ---
\makeglossaries
 
% ---
% entradas do glossario
% ---
 \newglossaryentry{pai}{
                name={pai},
                plural={pai},
                description={este é uma entrada pai, que possui outras
                subentradas.} }

 \newglossaryentry{componente}{
                name={componente},
                plural={componentes},
                parent=pai,
                description={descriação da entrada componente.} }
 
 \newglossaryentry{filho}{
                name={filho},
                plural={filhos},
                parent=pai,
                description={isto é uma entrada filha da entrada de nome
                \gls{pai}. Trata-se de uma entrada irmã da entrada
                \gls{componente}.} }
 
\newglossaryentry{equilibrio}{
                name={equilíbrio da configuração},
                see=[veja também]{componente},
                description={consistência entre os \glspl{componente}}
                }

\newglossaryentry{latex}{
                name={LaTeX},
                description={ferramenta de computador para autoria de
                documentos criada por D. E. Knuth} }

\newglossaryentry{abntex2}{
                name={abnTeX2},
                see=latex,
                description={suíte para LaTeX que atende os requisitos das
                normas da ABNT para elaboração de documentos técnicos e científicos brasileiros} }
% ---

% ---
% Exemplo de configurações do glossairo
\renewcommand*{\glsseeformat}[3][\seename]{\textit{#1}  
 \glsseelist{#2}}
% ---
              
                
% ----
% Início do documento
% ----
\begin{document}
 
% Retira espaço extra obsoleto entre as frases.
\frenchspacing
 
% ----------------------------------------------------------
% ELEMENTOS PRÉ-TEXTUAIS
% ----------------------------------------------------------
 
% ---
% Capa
% ---
\imprimircapa
% ---
 
% ---
% inserir o sumario
% ---
\pdfbookmark[0]{\contentsname}{toc}
\tableofcontents*
\cleardoublepage
% ---
  
 
% ----------------------------------------------------------
% ELEMENTOS TEXTUAIS
% ----------------------------------------------------------
\textual
 
% ----------------------------------------------------------
% Introdução
% ----------------------------------------------------------
\chapter*[Introdução]{Introdução}
\addcontentsline{toc}{chapter}{Introdução}
 
Este documento e seu código-fonte são exemplos de referência de uso da classe
\textsf{abntex2} e do pacote \textsf{abntex2cite}. O documento não se trata de
um modelo específico baseado em alguma norma da ABNT, mas exemplifica a
elaboração do elemento ``Glossário'' previsto nas normas ABNT NBR
10719:2011, ABNT NBR 14724:2011 e ABNT NBR 15287:2011, por exemplo. Este
documento deve ser utilizado como complemento aos demais exemplos de uso de
\abnTeX.

A expressão ``Modelo Canônico'' é utilizada para indicar que \abnTeX\ não é
modelo específico de nenhuma universidade ou instituição, mas que implementa tão
somente os requisitos das normas da ABNT. Uma lista completa das normas
observadas pelo \abnTeX\ é apresentada em \citeonline{abntex2classe}.
 
Sinta-se convidado a participar do projeto \abnTeX! Acesse o site do projeto em
\url{http://abntex2.googlecode.com/}. Também fique livre para conhecer,
estudar, alterar e redistribuir o trabalho do \abnTeX, desde que os arquivos
modificados tenham seus nomes alterados e que os créditos sejam dados aos
autores originais, nos termos da ``The \LaTeX\ Project Public
License''\footnote{\url{http://www.latex-project.org/lppl.txt}}.
 
Encorajamos que sejam realizadas customizações específicas deste exemplo para
universidades e outras instituições --- como capas, folha de aprovação, etc.
Porém, recomendamos que ao invés de se alterar diretamente os arquivos do
\abnTeX, distribua-se arquivos com as respectivas customizações.
Isso permite que futuras versões do \abnTeX~não se tornem automaticamente
incompatíveis com as customizações promovidas. Consulte
\citeonline{abntex2-wiki-como-customizar} par mais informações.
 
Este documento deve ser utilizado como complemento dos manuais do \abnTeX\
\cite{abntex2classe,abntex2cite,abntex2cite-alf} e da classe \textsf{memoir}
\cite{memoir}.
 
Equipe \abnTeX
  
Lauro César Araujo
 
 
% ---
% Capitulo que faz uso de elementos do glossario
% ---
\chapter{Orientações a respeito de glossários}
 
% ---
\section{Usar o glossário no texto}
% ---
 
Você pode definir as entradas do glossário no início do texto. Recomenda-se o
uso de um arquivo separado a ser inserido ainda no preâmbulo. Veja orientações
sobre inclusão de arquivos na \autoref{sec-include}.

No decorrer do texto, use os termos do glossário como na frase:

\begin{citacao}
Esta frase usa a palavra \gls{componente} e o plural de \glspl{filho}, ambas
definidas no glossário como filhas da entrada \gls{pai}. \Gls{equilibrio}
exemplifica o uso de um termo no início da frase. O software \gls{abntex2} é
escrito em \gls{latex}, que é definido no glossário como
\emph{`\glsdesc*{latex}'}.
\end{citacao}


A frase acima foi produzida com:

\begin{verbatim}
Esta frase usa a palavra \gls{componente} e o plural de \glspl{filho}, ambas
definidas no glossário como filhas da entrada \gls{pai}. \Gls{equilibrio}
exemplifica o uso de um termo no início da frase. O software \gls{abntex2} é
escrito em \gls{latex}, que é definido no glossário como
\emph{`\glsdesc*{latex}'}.
\end{verbatim}

Opcionalmente, incorpore todas as palavras do glossário de uma única vez ao
documento com o comando:

\begin{verbatim}
   \glsaddall
\end{verbatim}
 
A impressão efetiva do glossário é dada com:

\begin{verbatim}
   \printglossaries
\end{verbatim}

A impressão do glossário incorpora o número das páginas em que as entradas foram
citadas. Isso pode ser removido adicionando-se a opção \texttt{nonumberlist} em:

\begin{verbatim}
\usepackage[nonumberlist,style=index]{glossaries}%
\end{verbatim}

% ---
\section{Compilar um documento com glossário}
\label{sec-compilar-glossario}
% ---
 
Para compilar um documento \LaTeX\ com glossário use:

\begin{verbatim}
   pdflatex ARQUIVO_PRINCIPAL.tex
   bibtex ARQUIVO_PRINCIPAL.aux
   makeindex ARQUIVO_PRINCIPAL.idx 
   makeindex ARQUIVO_PRINCIPAL.nlo -s nomencl.ist -o ARQUIVO_PRINCIPAL.nls
   makeglossaries ARQUIVO_PRINCIPAL.aux
   pdflatex ARQUIVO_PRINCIPAL.tex
   pdflatex ARQUIVO_PRINCIPAL.tex
\end{verbatim}
 
O comando \texttt{makeglossaries} é um aplicativo Perl instalado
automaticamente pelas distribuições MacTeX, TeX Live e MiKTeX. Geralmente
usuários de Linux e de Mac OS X já possuem o interpretador Perl\footnote{O Perl
é uma linguagem de programação de scripts muito utilizada pela comunidade de
software livre. Veja o site do projeto em \url{http://www.perl.org/}.} instalado
e configurado e nenhuma configuração adicional é necessária.

Usuários de Windows, por outro lado, precisam instalar a ferramenta Perl para
que seja possível usar \texttt{makeglossaries}. Por sorte isso é simples. Para
obter a instalação do Perl para seu sistema operacional visite \url{http://www.perl.org/get.html}.

Alternativamente ao aplicativo Perl \texttt{makeglossaries}, é possível usar o
aplicativo \texttt{makeglossariesgui}\footnote{O título do aplicativo no CTAN
é \textit{Java GUI alternative to makeglossarires script}.}, que possui uma
interface gráfica baseada em Java. Para isso, consulte
\url{http://www.ctan.org/pkg/makeglossariesgui}. Funciona em Windows,
Linux e Mac OS X.

% ---
\section{Configuração de glossários}
% ---

O pacote \textsf{glossaries}, usado na produção dos glossários deste exemplo,
possui diversas configurações. É possível alterar o estilo da impressão do
glossário, criar campos adicionais, usar diversos glossários em
arquivos separados. Para isso e outras informações, consulte a documentação do
pacote \textsf{glossaries}: \url{http://www.ctan.org/pkg/glossaries}.

Consulte também o livro da WikiBooks sobre a produção de glossários:
\url{http://en.wikibooks.org/wiki/LaTeX/Glossary}.
 

\subsection{Estilos do glossário}

O pacote \textsf{glossaries} traz dezenas de estilos pré-definidos de
glossários. Eles estão disponíveis no capítulo 15 do manual do pacote
\cite{talbot2012}. O capítulo 16 contém instruções sobre como criar um estilo
personalizado.

Os estilos podem ser alterados com:

\begin{verbatim}
   \setglossarystyle{altlisthypergroup}
\end{verbatim}

O estilo \texttt{index} é ideal para construção de glossários com diversos
níveis hierárquicos do tipo pai-filho. Já o modelo \texttt{altlisthypergroup} é
mais adequado para glossários sem hierarquias. Teste também o modelo
\texttt{tree}.

Se desejar um único estilo de glossário padrão no documento, alternativamente
inclua a opção \texttt{style} nas opções da classe, do
seguinte modo:

\begin{verbatim}
   \usepackage[style=index]{glossaries}
\end{verbatim}

% ---
\section{Problemas com a ordem das palavras?}
% ---

Este exemplo do \abnTeX\ utiliza a ferramenta \texttt{makeindex} -- padrão das
distribuições \LaTeX\ mais comuns -- para ordenar as entradas do glossário.
Porém, essa ferramenta não possui opções de \textit{collation} e não funciona
bem para palavras escritas em idiomas que não sejam inglês.
Por isso, pode acontecer que letras acentuadas e outros caracteres
internacionais sejam ordenados de forma incorreta, como no exemplo (palavras não
necessariamente presentes na língua portuguesa):

\begin{alineas}
 \item Amor: ...
 \item Aviar: ...
 \item Avião: ...
 \item Aço: ...
\end{alineas}

Por sorte, é possível substituir o uso do \texttt{makeindex}
pelo \texttt{xindy}\footnote{\url{http://www.xindy.org/}}. Para isso, faça o
seguinte:

\begin{alineas}
  \item Certifique-se de que o Xindy esteja instalado. Em um terminal, digite:
  \texttt{xindy --version}\footnote{Caso o Xindy não esteja presente no sistema, é necessário
    instalá-lo. Usuários Linux Debian/Ubuntu podem usar: \texttt{sudo
    apt-get install xindy}. Usuários Windows e Mac podem acessar a página do
    Xindy, baixá-lo e instalá-lo.};
  \item No código LaTeX, ainda no preâmbulo, inclua a seguinte opção ao pacote glossaries:
  \begin{verbatim}
  \usepackage[xindy={language=portuguese},nonumberlist=true]{glossaries}
  \end{verbatim}
  \item Compile o glossário normalmente, conforme a
  \autoref{sec-compilar-glossario}.
\end{alineas}

% ----------------------------------------------------------
% Capitulo com exemplos de comandos inseridos de arquivo externo 
% ----------------------------------------------------------

\include{abntex2-modelo-include-comandos}
 
 
% ----------------------------------------------------------
% ELEMENTOS PÓS-TEXTUAIS
% ----------------------------------------------------------
\postextual
 
 
% ----------------------------------------------------------
% Referências bibliográficas
% ----------------------------------------------------------
\bibliography{abntex2-modelo-references}
 
% ----------------------------------------------------------
% Glossário
% ----------------------------------------------------------

% ---
% Define nome e preâmbulo do glossário
% ---
\phantompart
\renewcommand{\glossaryname}{Glossário}
\renewcommand{\glossarypreamble}{Esta é a descrição do glossário. Experimente
visualizar outros estilos de glossários, como o \texttt{altlisthypergroup},
por exemplo.\\
\\}

% ---
% Traduções para o ambiente glossaries
% ---
\providetranslation{Glossary}{Glossário}
\providetranslation{Acronyms}{Siglas}
\providetranslation{Notation (glossaries)}{Notação}
\providetranslation{Description (glossaries)}{Descrição}
\providetranslation{Symbol (glossaries)}{Símbolo}
\providetranslation{Page List (glossaries)}{Lista de Páginas}
\providetranslation{Symbols (glossaries)}{Símbolos}
\providetranslation{Numbers (glossaries)}{Números} 
% ---

% ---
% Estilo de glossário
% ---
% \setglossarystyle{index}
% \setglossarystyle{altlisthypergroup}
 \setglossarystyle{tree}


% ---
% Imprime o glossário
% ---
\cleardoublepage
\phantomsection
\addcontentsline{toc}{chapter}{\glossaryname}
\printglossaries
% ---
  
 
% ----------------------------------------------------------
% Apêndices
% ----------------------------------------------------------
 
% ----------------------------------------------------------
% Anexos
% ----------------------------------------------------------
 
%---------------------------------------------------------------------
% INDICE REMISSIVO
%---------------------------------------------------------------------
 
\end{document}
